%!TEX root = ../dokumentation.tex
\chapter{Fazit}
Petrinetze erlauben es, Systeme formal zu modellieren. Die Stärken von Petrinetzen liegen jedoch vor allem bei der Analyse und Verifikation der Modelle. Die Analyse auf Platzinvarianten liefert Aufschlüsse über die Funktionsweise eines Petrinetzes. Diese kann man zum Beispiel dafür verwenden, das Modell zu verifizieren.
Die Schwierigkeit der Petrinetze liegt bei der Modellierung.
Komplexe Systeme werden schnell unübersichtlich und können von Menschen kaum noch modelliert werden.
Abhilfe schaffen hierfür die high-level Petrinetze, die in low-level Petrinetze umgewandelt werden können. Außerdem lassen sich viele konkretere Modellierungssprachen direkt auf Petrinetze zurückführen. Dazu zählen zum Beispiel BPMN 2.0 und EPK\footnote{\cite{fernuni_hagen:diss}}, zwei Sprachen zur Modellierung von Geschäftsprozessen. Entsprechende Software kann die modellierten Geschäftsprozesse dann intern in Petrinetze umwandeln. So kann dem Benutzer eine einfache Möglichkeit zur Modellierung geboten werden. Gleichzeitig können mächtige Funktionen zur Analyse, Verifikation und Simulation, die auf den theoretischen Erkenntnissen über Petrinetze aufbauen, implementiert werden.
Zusammenfassend lässt sich sagen, dass Petrinetze ein mächtiges Werkzeug sind. Im Geschäftsprozessmanagement wird ein Anwender aufgrund ihrer Komplexität, jedoch eher selten direkt mit ihnen in Berührung kommen.