%!TEX root = ../dokumentation.tex
\chapter{Theorie}

%TODO Hier fehlt noch ein einführender Text zwischen der Über- und der Unterüberschrift

\section{Aufbau von Petrinetzen} % (fold)
\label{sub:aufbau_von_petrinetzen}
	Im Prinzip können Petrinetze als Graphen aufgefasst werden.
	Diese Graphen besitzen allerdings besondere Knoten, die als Komponenten bezeichnet werden.
	Im folgenden werden die Komponenten eines Petrinetzes beschrieben.

	\paragraph{Plätze} sind die passiven Komponenten eines Petrinetzes. Sie dienen repräsentieren Zustände oder lagern Objekte.
	Ein Platz wird durch einen Kreis dargestellt:
	\begin{figure}[h]
		\centering
		\begin{tikzpicture}
			\node[place](emptyPlace){};
			\node[below=0.2cm of emptyPlace](emptyPlaceLabel){Leerer Platz};
			\node[right= of emptyPlaceLabel](filledPlaceLabel){Gefüllter Platz};
			\node[place, tokens=1, above=0.2cm of filledPlaceLabel](filledPlace){};
		\end{tikzpicture}
		\caption{Beispiel für einen leeren Platz und einen Platz mit einer (abstrakten) Marke.}
		\label{fig:platz}
	\end{figure}

	Die Objekte, die von Plätzen gelagert werden, werden allgemein als \textbf{Marken} bezeichnet. Grundsätzlich kann ein Platz beliebig viele Marken enthalten. Außerdem können in einem Modell unterschiedliche Typen von Marken verwendet werden. Eine Marke kann ein konkretes Objekt aus der realen Welt repräsentieren (Maschinen, Werkstücke, Menschen, ...). Es können jedoch auch abstrakte Marken verwendet werden, die dann meist als schwarzer Punkt dargestellt werden, wie in Abbildung~\ref{fig:platz} zu sehen. Abstrakte Marken sind sehr nützlich, um Zustände, in dem sich das System befinden kann, zu modellieren.

	\paragraph{Transitionen} sind die aktiven Komponenten eines Petrinetzes.
    Sie beschreiben eine elementaren Aktion, die Dinge erzeugen, transportieren, verändern oder vernichten \footnote{\cite{hu_berlin:petrinetze:plaetze}}.
	Eine Transition wird durch ein Quadrat dargestellt:
	\begin{figure}[h]
		\centering
		\begin{tikzpicture}
			\node[transition]{};
		\end{tikzpicture}
		\caption{Beispiel für eine Transition}
		\label{fig:transition}
	\end{figure}

	Transitionen können zusätzlich noch mit einer Formel beschriftet werden. Diese ist eine Vorbedingung, die erfüllt sein muss, damit die Transition aktiviert werden kann. Wann genau eine Transition aktiviert ist und was das bedeutet, wird später in Kapitel~\ref{ssub:ablaufregeln} noch diskutiert.

	\paragraph{Kanten} sind selbst keine Komponenten, sondern stellen Beziehungen zwischen Komponenten her. Kanten werden durch einen Pfeil gekennzeichnet und verbinden immer eine Transition mit einem Platz oder einen Platz mit einer Transition. Niemals jedoch verbinden sie zwei Plätze oder zwei Transitionen.

	Kanten, die von einer Transition auf eine Stelle zeigen, drücken aus, dass diese Transition Objekte erzeugt und sie in der Stelle ablegt. Kanten, die von einer Stelle auf eine Transition zeigen, drücken aus, dass diese Transition Objekte aus der Stelle entnimmt.

	Eine Kante kann zusätzlich noch mit einem Term beschriftet werden. Das Ergebnis dieses Terms müssen Marken sein. Damit wird angegeben, wie viele und welche Marken die Transition aus der Stelle entnimmt beziehungsweise in ihr erzeugt. Im Normalfall ist die Beschriftung einfach ein konstanter Term, es sind jedoch auch Terme mit beliebigen Variablen erlaubt.

	\begin{figure}[ht]
		\centering
		\begin{tikzpicture}
			\node[place, tokens=1](konsStart){};
			\node[transition, right=3cm of konsStart](konsEnd){}
				edge[pre] node[token,below=1pt,label=above:konsumiert]{} (konsStart);

			\node[transition, below= of konsStart](prodStart){};
			\node[place, tokens=1, right=3cm of prodStart](prodEnd){}
				edge [pre] node[token,below=1pt,label=above:produziert]{} (prodStart);
		\end{tikzpicture}
		\caption{Beispiele für Kanten}
		\label{fig:kanten}
	\end{figure}

\section{Ablaufregeln}
\label{ssub:ablaufregeln}
	Der allgemeine Aufbau von Petrinetzen ist nun bekannt. Nach welchen Regeln sich ein Petrinetz genau verhält, wurde jedoch noch nicht festgelegt. In den folgenden Abschnitten sollen diese Regeln erläutert werden.
	\\\\
	Als \textbf{Markierung} eines Petrinetzes bezeichnet man die Verteilung von Marken auf die Plätze. In Abbildung~\ref{fig:markierung1}~und~\ref{fig:markierung2} sind zwei unterschiedliche Markierungen eines Petrinetzes dargestellt. Ein Petrinetz zusammen mit einer Anfangsmarkierung, in der sich das Petrinetz zu Beginn befindet, wird Systemnetz genannt.
	\begin{figure}[t]
		\centering
		\begin{tikzpicture}
			\node[transition,label=left:Produkt herstellen](t){};

			\node[place, below left=of t, label=below:Auftrag liegt vor, tokens=2](A){};
			\node[place, above left=of t, label=Lagerbestand](M){}
				[children are tokens]
				child{node[draw,regular polygon,regular polygon sides=3,inner sep=1pt]{}}
				child{node[draw,diamond,inner sep=1.4pt]{}}
				child{node[draw,diamond,inner sep=1.4pt]{}};
			\node[place, right=of t, label=Endprodukt](P){};

			\draw (t)	edge[pre]	node[token,left=2pt]{}	(A)
						edge[pre]	node[draw,regular polygon,regular polygon sides=3,inner sep=1pt,left=3pt]{}	(M)
						edge[post]	node[draw, below=1pt]{}	(P);
		\end{tikzpicture}
		\caption{Markierung 1 eines Petrinetzes}
		\label{fig:markierung1}
	\end{figure}
	\begin{figure}[t]
		\centering
		\begin{tikzpicture}
			\node[transition,label=left:Produkt herstellen](t){};

			\node[place, below left=of t, label=below:Auftrag liegt vor,tokens=1](A){};
			\node[place, above left=of t, label=Lagerbestand](M){}
				[children are tokens]
				child{node[draw,diamond,inner sep=1.4pt]{}}
				child{node[draw,diamond,inner sep=1.4pt]{}};

			\node[place, right=of t, label=Endprodukt](P){}
				[children are tokens]
				child{node[draw]{}};

			\draw (t)	edge[pre]	node[token,left=2pt]{}	(A)
						edge[pre]	node[draw,regular polygon,regular polygon sides=3,inner sep=1pt,left=3pt]{}	(M)
						edge[post]	node[draw, below=1pt]{}	(P);
		\end{tikzpicture}
		\caption{Markierung 2 eines Petrinetzes}
		\label{fig:markierung2}
	\end{figure}
	\\\\
	Als nächstes wird erklärt, wann man eine Transition als \textbf{aktiviert} bezeichnet. Anschaulich gesehen ist eine Transition $t$ in einer bestimmten Markierung aktiviert, wenn alle Plätze, aus denen $t$ Marken entnimmt, dafür genug Marken beinhalten.
	Formal muss also jeder Platz, von dem eine Kante auf $t$ zeigt, von jedem Markentyp genausoviele oder mehr Marken beinhalten als in der Beschriftung der Kante angegeben.

	Im Petrinetz aus Abbildung~\ref{fig:markierung1} entnimmt die Transition $Produkt herstellen$ eine dreieckige Marke aus dem Platz $Lagerbestand$ und eine schwarze Marke aus $Auftrag liegt vor$. Beide Plätze enthalten die geforderten Marken. Also ist die Transition aktiviert.

	In Abbildung~\ref{fig:markierung2} ist das selbe Petrinetz in einer anderen Markierung dargestellt. Hier enthält $Auftrag liegt vor$ genug schwarze Marken. Aus dem Platz $Lagerbestand$ entnimmt die Transition jedoch eine dreieckige Marke, aber in dem Platz sind nur Rautenförmige Marken enthalten. Daher ist die Transition in Markierung 2 nicht aktiviert.

	Diese Vorgehensweise funktioniert für Kanten, die mit konstanten Termen beschriftet sind. Wenn im Term einer Kante Variablen vorkommen, muss zunächst eine Belegung der Variablen mit echten Werten gefunden werden, für die diese Bedingung zutrifft. Außerdem muss die Formel in der Transition für die Variablenbelegung wahr sein. Wenn eine passende Variablenbelegung existiert, dann ist die Transition in der entsprechenden Markierung unter dieser Variablenbelegung aktiviert. Andernfalls ist die Transition nicht aktiviert.

	\begin{figure}[h]
		\centering
		\begin{tikzpicture}
			\node[place,label=below:Platz 1](p1){}
				[children are tokens]
				child{node[token]{}}
				child{node[draw,fill]{}};
			\node[place,label=below:Platz 2,below=of p1](p2){}
				[children are tokens]
				child{node[token]{}}
				child{node[draw,fill]{}};
			\node[place,label=below:Platz 3,below=of p2,tokens=1](p3){};

			\node[transition,right=3cm of p2](t){}
				edge[pre,bend right]	node[right]{X}		(p1)
				edge[pre,bend right]	node[below]{X}		(p2)
				edge[post,bend left]	node[below]{X,Y}	(p2)
				edge[pre,bend left]		node[below]{Y}		(p3);

			\node[place,right=of t,label=below:Platz 4](pr){}
				edge[pre] node[below]{X} (t);
		\end{tikzpicture}
		\caption{Ein Petrinetz mit Variablen in den Kantenbeschriftungen}
		\label{fig:varkanten1}
	\end{figure}

	Abbildung~\ref{fig:varkanten1} zeigt ein Petrinetz mit Variablen in den Kantenbeschriftungen. In der gezeigten Markierung gibt es zwei Variablenbelegungen, für die die Transition aktiv ist:
	\begin{itemize}
		\item $X$ = \tikz{\node[fill]{};}	\hspace{1cm} $Y$ = \tikz{\node[token]{};} 
		\item $X$ = \tikz{\node[token]{};}	\hspace{1cm} $Y$ = \tikz{\node[token]{};}
	\end{itemize}
	Eine Variablenbelegung mit $Y$ = \tikz{\node[fill]{};}, bei der die Transition aktiviert ist, kann es in der gezeigten Markierung nicht geben, denn in $Platz 3$ ist ausschließlich eine kreisförmige Marke vorhanden.
	
	Wenn eine Transition aktiviert ist, dann kann sie \textbf{eintreten}. Beim Ablauf eines Petrinetzes kann in einer Markierung immer nur eine Transition eintreten. Welche der aktivierten Transitionen eintritt, ist nicht deterministisch bestimmt. Durch das Eintreten einer Transition wird eine neue Markierung für das Petrinetz erzeugt. Dabei werden aus allen Plätzen, von denen eine Kante auf die Transition zeigen, die in der Kantenbeschriftung angegebenen Marken entfernt. Außerdem werden zu allen Plätzen, auf die eine Kante zeigt, die bei der Transition beginnt, die in der Kantenbeschriftung angegebenen Marken hinzugefügt.

	Wenn die Kantenbeschriftungen Variablen enthalten, müssen zunächst die Werte aus der Variablenbelegung, für die die Transition aktiv ist, eingesetzt werden. 

	Wenn die Transition $Produkt herstellen$ in Abbildung~\ref{fig:markierung1} eintritt, dann wird aus dem Platz $Lagerbestand$ ein Marke \tikz{\node[draw,regular polygon,regular polygon sides=3,inner sep=1pt]{};} entfernt und aus dem Platz $Auftrag liegt vor$ wird eine Marke \tikz{\node[token]{};} entfernt. Außerdem wird zum Platz $Endprodukt$ eine Marke \tikz{\node[draw]{};} hinzugefügt. So entsteht die Markierung, die in Abbildung~\ref{fig:markierung2} zu sehen ist.

	Wie oben bereits erläutert, ist die Transition aus Abbildung~\ref{fig:varkanten1} für die Variablenbelegung $X$ = \tikz{\node[fill]{};}, $Y$ = \tikz{\node[token]{};} aktiviert. Wenn die Transition unter dieser Variablenbelegung eintritt, wird aus $Platz 3$ eine Marke \tikz{\node[token]{};} und aus $Platz 1$ eine Marke \tikz{\node[fill]{};} entfernt. Aus $Platz 2$ wird zunächst ebenfalls eine Marke \tikz{\node[fill]{};} entfernt und die Marken \tikz{\node[fill]{};} und \tikz{\node[token]{};} wieder hinzugefügt. Außerdem wird zu $Platz 4$ eine Marke \tikz{\node[fill]{};} hinzugefügt.

	\begin{figure}[h]
		\centering
		\begin{tikzpicture}
			\node[place,label=below:Platz 1,tokens=1](p1){};
			\node[place,label=below:Platz 2,below=of p1](p2){}
				[children are tokens]
				child{node[token]{}}
				child{node[token]{}}
				child{node[draw,fill]{}};
			\node[place,label=below:Platz 3,below=of p2](p3){};

			\node[transition,right=3cm of p2](t){}
				edge[pre,bend right]	node[right]{X}		(p1)
				edge[pre,bend right]	node[below]{X}		(p2)
				edge[post,bend left]	node[below]{X,Y}	(p2)
				edge[pre,bend left]		node[below]{Y}		(p3);

			\node[place,right=of t,label=below:Platz 4](pr){}
				[children are tokens]
				child{node[draw,fill]{}};
				
			\draw (pr) edge[pre] node[below]{X} (t);
		\end{tikzpicture}
		\caption{Das Petrinetz aus Abbildung~\ref{fig:varkanten1} nach dem Eintreten der Transition.}
		\label{fig:varkanten2}
	\end{figure}

	Ein weiteres Beispiel für den Ablauf eines Systemnetzes sind \textbf{Zähler}. Diese erleichtern unter anderem die Modellierung von Lagern mit begrenzten Stückzahlen. Der Zähler, der in Abbildung~\ref{fig:zaehler} zu sehen ist, besteht aus einer Transition, die maximal 10 mal eintreten kann. Als Marke in $Platz 1$ wird eine natürliche Zahl verwendet. Bei jedem Eintritt der Transition wird diese in Form der Variable $x$ aus dem Platz entnommen und dafür die Zahl $x-1$ wieder hinzugefügt. So verringert sich die Zahl in dem Platz bei jedem Eintreten der Transition um 1. Die Transition hat außerdem als Vorbedingung, dass $x > 0$ gelten muss. Deshalb kann sie insgesamt nur 10 mal eintreten.
	\begin{figure}[h]
		\centering
		\begin{tikzpicture}
			\node(left){};
			\node[right=3cm of left](right){};
			\node[transition, right=1cm of left, left=1cm of right](transition){$x>0$}
			edge[pre, dashed] (left)
			edge[post, dashed] (right);
			\node[place, below= of transition,label=below:Platz 1] {10}
			edge[pre, anchor=east, transform canvas={xshift=-1mm}] node {$x$} (transition)
			edge[post, anchor=west, transform canvas={xshift=1mm}] node {$x-1$} (transition);
		\end{tikzpicture}
		\caption{Ein Zähler}
		\label{fig:zaehler}
	\end{figure}
\section{Weitere Ausdrucksmöglichkeiten}
	% @:    so sagen?
	\paragraph{Schnittstellen} ermöglichen auf Aktivitäten aus der Umgebung einzugehen.
	Mit Hilfe der bisherig vorgestellten Bestandteile lassen sich bereits geschlossene Systeme modellieren.
	Sollen allerdings Interaktionen eines Geschäftsprozesses mit z.B. einem Kunden modelliert werden, benötigt man Schnittstellen.
	Diese werden durch eine Transition dargestellt.

	\paragraph{Kalte Transitionen} dienen der Modellierung von Transitionen, bei denen nicht klar ist ob sie eintreffen.
	Wird z.B. ein Prozess von einem Kunden durch dessen Bezahlung ausgelöst, ist im Vorhinein nicht klar ob dieses Ereignis jemals eintreffen wird.
	Ist das Gegenteil der Fall wird die Transition als \enquote{warm} bezeichnet.
	Kalte Transitionen werden durch eine Transition dargestellt, die ein $\epsilon$ beinhalten dargestellt.
	\begin{center}
		\begin{tikzpicture}
			\node[transition]{$\epsilon$};
		\end{tikzpicture}
	\end{center}

\section{Variationen}
	Mit den zunehmenden Anforderungen an Petrinetze sind zahlreiche Variationen von Petrinetzen entstanden.
	Dabei werden die Varianten in zwei Kategorien gegliedert:
	\begin{itemize}
	    \item low-level Petrinetze
	    \item high-level Petrinetze
	\end{itemize}

	\subsection{Elementare Petrinetze}
		Häufig werden Petri Netze dazu verwendet darzustellen, wo ein Kontrollfluss aktuell steht. \footnote{(vgl. \cite[Seite 39]{hu_berlin:petrinetze})}
		Will man einen Prozess genauer analysieren geht dies am einfachsten in dem man den Prozess abstrakt darstellt.
		Dafür bieten sich elementare Petrinetze an, die auch als elementare Systemnetze bezeichnet werden.\\
		Bei einem elementaren Petrinetz werden die Marken auf den Kanten weggelassen. Außerdem gibt es nur eine Form von Marken, die in der Regel durch einen schwarzen Punkt \textbullet visualisiert wird.
		Dabei beschränkt sich ein elementares Petrinetz auf Transitionen, Plätze und Marken.
		Aufgrund der abstrakten Modellierung und geringen Anzahl an Komponenten gehören elementare Petrinetze zu den low-level Petrinetzen.

	\subsection{Gefärbte Petrinetze}
		Gefärbte Petrinetze werden hauptsächlich zur Modellierung komplexer Systeme verwenden, da Sie zu der Gruppe der high-level Petrinetzen gehören.
		Im Unterschied zu anderen Petrinetzen sind die gefärbten Petrinetze wesentlich übersichtlicher, können allerdings jeder Zeit wieder in ein low-level Petrinetz zurückgeführt werden.
		Dabei bieten die gefärbten Petrinetze weitere Funktionalitäten wie z.B.:
		\begin{itemize}
		    \item Typen
		    \item Knotenfunktionen
		    \item Typfunktionen
		    \item Wächterfunktionen
		    \item Kantenausdrucksfunktionen
		    \item Initialisierungsfunktionen
		\end{itemize}
		\footnote{(vgl.: \cite{tu_dresden:petrinetze})}
		auf die in diesem Dokument nicht weiter eingegangen wird.

\section{Analyse und Verifikation mit Platzinvarianten}
	Es gibt verschiedene Möglichkeiten zur Analyse von Petrinetzen. Grundsätzlich kann man dabei zwischen der Analyse von Zustands- und Ablaufeigenschaften unterscheiden. Bei ersterem untersucht man Eigenschaften, die allgemein für alle im Systemnetz erreichbaren Markierungen gelten. Ablaufeigenschaften wiederum beschreiben Eigenschaften von einzelnen Markierungen, welche nach mehreren Schritten aus anderen Markierungen hervorgehen können.

	In den folgenden Abschnitten wird exemplarisch die Bestimmung von Platzinvarianten in elementaren Systemnetzen beschrieben. Dabei geht es darum, Gleichungen zu finden, die für jede erreichbare Markierung gelten. Diese kann man dann beispielsweise für die Verifikation des Modells verwenden.

	\subsection{Einführung}
	\label{sub:platzinvarianten_einf}
		Oft werden in Petrinetzen Material- oder Informationsflüsse modelliert. Dabei wird nichts erzeugt oder verbraucht, sondern das Material oder die Informationen werden bloß transportiert. Äquivalent dazu werden auch die entsprechenden Marken im Petrinetz nicht erzeugt oder zerstört, sondern nur von Platz zu Platz verschoben. Ein Beispiel ist in Abbildung~\ref{fig:exinvariantsimple} zu sehen. Die Gesamtzahl der Marken bleibt hier immer gleich. Wenn man mit $|Platz|$ die Anzahl der Marken im gegebenen Platz bezeichnet, kann man bei diesem Beispiel sagen, dass die Gleichung

		$$|P_1| + |P_2| + |P_3| = 5$$

		immer wahr ist.	Ziel der Platzinvarianten ist es solche Gleichungen auch in größeren Petrinetzen zu finden.

		\begin{figure}[h]
			\centering
			\begin{tikzpicture}
				\node[place,label=below:$P_1$,tokens=5](p1){};
				\node[transition,label=below:$P_1$ nach $P_2$,right=2cm of p1](t1){}
					edge[pre](p1);
				\node[place,label=below:$P_2$,right=2cm of t1](p2){}
					edge[pre](t1);
				\node[transition,label=below:$P_2$ nach $P_3$,right=2cm of p2](t2){}
					edge[pre](p2);
				\node[place,label=below:$P_3$,right=2cm of t2](p3){}
					edge[pre](t2);
			\end{tikzpicture}
			\caption{Ein einfaches Petrinetz mit einer Platzinvariante.}
			\label{fig:exinvariantsimple}
		\end{figure}
		
		Platzinvarianten sind jedoch noch etwas allgemeiner anwendbar. In Abbildung~\ref{fig:exinvariant} ist ein einfaches Systemnetz zum Verpacken von Produkten dargestellt. Auch hier werden wieder Materialflüsse dargestellt. Dabei werden Produkte mit Verpackungen zu verpackten Produkten kombiniert. Auch wenn so kein Material erzeugt oder zerstört wird, bleibt die Anzahl der Marken hier nicht gleich. Trotzdem lässt sich das Petrinetz mit einer Platzinvariante beschreiben.

		\begin{figure}[h]
			\centering
			\begin{tikzpicture}
				\node[transition,label=left:verpacken](t){};
				\node[place,label=above:Produkte,above=of t,tokens=4](p1){}
					edge[post](t);
				\node[place,label=below:Verpackungen,below=of t,tokens=3](p2){}
					edge[post](t);
				\node[place,label=below:Verpackte Produkte,right=2cm of t,tokens=2](p3){}
					edge[pre](t);
			\end{tikzpicture}
			\caption{Auch für Petrinetze wie dieses kann man Invarianten finden.}
			\label{fig:exinvariant}
		\end{figure}
		
		Eine Marke in $Verpackte Produkte$ ist nämlich genauso viel wert, wie eine Marke in $Produkte$ zusammen mit einer Marke aus $Verpackungen$. Um auszudrücken, dass eine Marke aus $Verpackte Produkte$ zwei Marken aus den anderen Plätzen entspricht, kann man die Anzahl der Marken in $Verpackte Produkte$ mit zwei multiplizieren. Dann gillt immer die Gleichung:

		$$2\cdot|Verpackte Produkte| + |Produkte| + |Verpackungen| = 11$$

	\subsection{Formale Definition}
	\label{sub:formale_definition}
		Gegeben sei ein Elementares Petrinetz mit $k$ Plätzen. Wenn man die Plätze als $P_1$,~$P_2$,~$P_3$,~...,~$P_k$ bezeichnet und jedem Platz $P_t$ eine Zahl $n_t$ zuordnet, dann hat die Gleichung einer Platzinvariante die Form:
		$$
		\sum_{s=1}^k n_s \cdot |P_s| = n_1 \cdot |P_1| + n_2 \cdot |P_2| + .... + n_k \cdot |P_k| = c
		$$
		Dabei wird c als die Konstante der Platzinvariante bezeichnet. 

		Die Zahlen $n_1$, $n_2$, ..., $n_t$ werden genau dann als Platzinvariante bezeichnet, wenn sich $c$ beim Eintreten einer beliebige Transition in einer beliebigen Markierung nicht ändert.

	\subsection{Platzinvarianten bestimmen}
		Die Platzinvarianten aus den beiden vorherigen Beispielen (Abbildung~\ref{fig:exinvariantsimple}~und~\ref{fig:exinvariant}) kann man mit etwas Übung schon beim Betrachten des Petrinetzes erkennen. Bei größeren Petrinetzen wird das jedoch schwieriger und fehleranfälliger. Deshalb soll im folgenden Abschnitt ein Verfahren zum Bestimmen von Platzinvarianten in beliebig komplexen Petrinetzen vorgestellt werden. Als Beispiel wird dabei das Petrinetz aus Abbildung~\ref{fig:invariantdemo} dienen.

		\begin{figure}[h]
			\centering
			\begin{tikzpicture}[scale=0.8, every node/.style={scale=0.8},node distance=0.8cm]
				\node[place,tokens=7,label=below:$P_1$](p1){};
				\node[transition,right=of p1,label=below:$T_1$](t1){};
				\node[place,right=of t1,label=below:$P_2$](p2){};
				\node[place,below=of p2,label=below:$P_3$](p3){};
				\node[place,above=of p2,label=below:$P_4$](p4){};
				\node[transition,right=of p2,label=below:$T_2$](t2){};
				\node[place,above right=of t2,label=below:$P_5$](p5){};
				\node[place,below right=of t2,label=below:$P_6$](p6){};
				\node[transition,right=of p6,label=below:$T_3$](t3){};
				\node[place,right=of t3,label=below:$P_7$](p7){};

				\draw (p1) edge[post] node[token]{} (t1);
				\draw (t1) edge[post] node[tokens=2]{} (p2);
				\draw (t1) edge[post] node[tokens=2]{} (p3);
				\draw (t1) edge[post] node[token]{} (p4);
				\draw (p2) edge[post] node[token]{} (t2);
				\draw (p3) edge[post] node[tokens=2]{} (t2);
				\draw (t2) edge[post] node[token]{} (p5);
				\draw (t2) edge[post] node[token]{} (p6);
				\draw (p5) edge[post] node[token]{} (t3);
				\draw (p6) edge[post] node[token]{} (t3);
				\draw (p4) edge[post] node[token]{} (t3);
				\draw (t3) edge[post,bend left=70] node[token]{} (p1);
				\draw (t3) edge[post] node[token]{} (p7);

			\end{tikzpicture}
			\caption{Ein Beispielnetz zum Bestimmen von Platzinvarianten}
			\label{fig:invariantdemo}
		\end{figure}

		Um die Notation zu vereinfachen, definieren wir uns zunächst die Hilfsfunktion $d(t,p)$. Diese soll die Veränderung beschreiben, die eine Transition an einem Platz durchführt, wenn sie eintritt. Wenn $t$ eine Transition ist und $p$ ein Platz, dann ist 

		$$d(t,p) := (\text{Anzahl der Marken, die t in p erzeugt}) - (\text{Anzahl der Marken, die t aus p entnimmt})$$
		
		Im Beispielpetrinetz aus Abbildung~\ref{fig:invariantdemo} lauten die Funktionswerte von $d$ wie folgt:

		\begin{center}		
			\begin{tabular}{ccc}
				$d(T_1,P_1) = -1$ 	& $d(T_2,P_1) = 0$ 		& $d(T_3,P_1) = 1$ 		\\
				$d(T_1,P_2) = 2$ 	& $d(T_2,P_2) = -1$ 	& $d(T_3,P_2) = 0$ 		\\
				$d(T_1,P_3) = 2$ 	& $d(T_2,P_3) = -2$ 	& $d(T_3,P_3) = 0$ 		\\
				$d(T_1,P_4) = 1$ 	& $d(T_2,P_4) = 0$ 		& $d(T_3,P_4) = -1$ 	\\
				$d(T_1,P_5) = 0$ 	& $d(T_2,P_5) = 1$ 		& $d(T_3,P_5) = -1$ 	\\
				$d(T_1,P_6) = 0$ 	& $d(T_2,P_6) = 1$ 		& $d(T_3,P_6) = -1$ 	\\
				$d(T_1,P_7) = 0$ 	& $d(T_2,P_7) = 0$ 		& $d(T_3,P_7) = 1$ 		\\
			\end{tabular}
		\end{center}

		Um eine Platzinvariante zu finden, müssen konkrete Werte für die Variablen $n_1$, $n_2$, ..., $n_t$ bestimmt werden. Diese müssen der Bedingung aus Abschnitt~\ref{sub:formale_definition} entsprechen: Beim Eintreten einer beliebigen Transition darf sich die Konstante $c$ der Schleifeninvariante nicht verändern. Die Veränderung von $c$ beim Eintreten einer Transition $t$ berechnet sich durch:
		$$
		\Delta c = \sum_{s=1}^k n_s \cdot d(t, P_s) = n_1 \cdot d(t, P_1) + n_2 \cdot d(t, P_2) + .... + n_k \cdot d(t, P_k)
		$$
		Indem man nun für jede Transition die Veränderung der Konstante mit $0$ gleichsetzt, erzwingt man, dass die Bedingung für die Platzinvariante eingehalten wird. Für jede Transition $t$ muss also gelten: 
		$$
		\sum_{s=1}^k n_s \cdot d(t, P_s) = n_1 \cdot d(t, P_1) + n_2 \cdot d(t, P_2) + .... + n_k \cdot d(t, P_k) = 0
		$$
		So entsteht ein lineares Gleichungssystem, das zum Beispiel mit dem Gaußschesn Eliminationsverfahren gelöst werden kann. Beim Petrinetz aus Abbildung~\ref{fig:invariantdemo} ergeben sich die folgenden Gleichungen:
		$$
		\arraycolsep=1.4pt
		\begin{vmatrix}
			T_1: & n_1 \cdot (-1) &+ & n_2 \cdot 2 &+ & n_3 \cdot 2 &+ & n_4 \cdot 1 &+ & n_5 \cdot 0 &+ & n_6 \cdot 0 &+ & n_7 \cdot 0 & = 0 \\
			T_2: & n_1 \cdot 0 &+ & n_2 \cdot (-1) &+ & n_3 \cdot (-2) &+ & n_4 \cdot 0 &+ & n_5 \cdot 1 &+ & n_6 \cdot 1 &+ & n_7 \cdot 0 & = 0 \\
			T_3: & n_1 \cdot 1 &+ & n_2 \cdot 0 &+ & n_3 \cdot 0 &+ & n_4 \cdot (-1) &+ & n_5 \cdot (-1) &+ & n_6 \cdot (-1) &+ & n_7 \cdot 1 & = 0 \\
		\end{vmatrix}
		$$
		Durch Lösen des Gleichungssystems kommt man auf die Werte für $n_1$,~$n_2$,~...,~$n_7$. Jede Lösung ist eine Platzinvariante des Petrinetzes. Das Gleichungssystem hat unendlich viele Lösungen, entsprechend gibt es auch unendlich viele Platzinvarianten. Einige exemplarische Lösungen lauten:
		$$
		\begin{matrix}
			\text{Platzinvariante 1}: & n_1 = 2 	& n_2 = 0 	& n_3 = 1 	& n_4 = 0 	& n_5 = 2 	& n_6 = 0 	& n_7 = 0 \\
			\text{Platzinvariante 2}: & n_1 = 1 	& n_2 = 0 	& n_3 = 0 	& n_4 = 1 	& n_5 = 0 	& n_6 = 0 	& n_7 = 0 \\
			\text{Platzinvariante 3}: & n_1 = 2 	& n_2 = 0 	& n_3 = 1 	& n_4 = 0 	& n_5 = 0 	& n_6 = 2 	& n_7 = 0 \\
			\text{Platzinvariante 4}: & n_1 = -2	& n_2 = -2 	& n_3 = 1 	& n_4 = 0 	& n_5 = 0 	& n_6 = 0 	& n_7 = 2 \\
		\end{matrix}
		$$
		Um die Gleichung der Platzinvarianten zu bestimmen, fehlen noch die Konstanten der Platzinvarianten. Diese können aus der Anfangsmarkierung nach der Formel aus Abschnitt~\ref{sub:formale_definition} berechnet werden. Für die oben bestimmten Platzinvarianten ergibt sich so:
		$$
		\begin{matrix}
			\text{Platzinvariante 1}: & c = 14 	\\
			\text{Platzinvariante 2}: & c = 7 	\\
			\text{Platzinvariante 3}: & c = 14 	\\
			\text{Platzinvariante 4}: & c = -14 \\
		\end{matrix}
		$$
		Anschließend können die Gleichungen der Platzinvarianten aufgeschrieben werden. Dafür müssen die Werte für $n_1$,~$n_2$,~...,~$n_k$ sowie für c in die Formel aus Abschnitt~\ref{sub:formale_definition} eingesetzt werden. So erhält man für das Systemnetz aus Abbildung~\ref{fig:invariantdemo} die folgenden Gleichungen:
		$$
		\begin{matrix}
			\text{Platzinvariante 1}: & 2 \cdot |P_1| + |P_3| + 2 \cdot |P_5| = 14 \\
			\text{Platzinvariante 2}: & |P_1| + |P_4| = 7 \\
			\text{Platzinvariante 3}: & 2 \cdot |P_1| + |P_3| + 2 \cdot |P_6| = 14 \\
			\text{Platzinvariante 4}: & (-2) \cdot |P_1| + (-2) \cdot |P_2| + |P_3| + 2 \cdot |P_7|= -14 \\
		\end{matrix}
		$$
		Diese Gleichungen gelten in allen erreichbaren Markierungen.

		Das gezeigte Verfahren ist universell einsetzbar und ermöglicht es, alle Platzinvarianten eines gegebenen Systemnetzes zu finden. Da es in der Hauptsache auf dem Lösen eines Linearen Gleichungssystems beruht, sollte es außerdem möglich sein, das Verfahren zu automatisieren. Wie man anhand der Gleichung der Platzinvariante 4 sieht, gehen die Ergebnisse der Analyse über die in Abschnitt~\ref{sub:platzinvarianten_einf} vorgestellte Analogie zu Material- oder Informationsflüssen hinaus, denn durch die negativen Koeffizienten lässt sich diese hier nicht anwenden. Welche Möglichkeiten die Gleichungen zum weiteren Vorgehen bieten, wird in Kapitel~\ref{chap:fallstudie}~-~\nameref{chap:fallstudie} besprochen. 
