%!TEX root = ../dokumentation.tex
\chapter{Theorie}

%TODO Hier fehlt noch ein einführender Text zwischen der Über- und der Unterüberschrift

\section{Aufbau von Petrinetzen} % (fold)
\label{sub:aufbau_von_petrinetzen}
	Im Prinzip können Petrinetze als Graphen aufgefasst werden.
	Diese Graphen besitzen allerdings besondere Knoten, die als Komponenten bezeichnet werden.
	Im folgenden werden die Komponenten eines Petrinetzes beschrieben.

	\paragraph{Plätze} sind die passiven Komponenten eines Petrinetzes. Sie dienen repräsentieren Zustände oder lagern Objekte.
	Ein Platz wird durch einen Kreis dargestellt:
	\begin{figure}[h]
		\centering
		\begin{tikzpicture}
			\node[place](emptyPlace){};
			\node[below=0.2cm of emptyPlace](emptyPlaceLabel){Leerer Platz};
			\node[right= of emptyPlaceLabel](filledPlaceLabel){Gefüllter Platz};
			\node[place, tokens=1, above=0.2cm of filledPlaceLabel](filledPlace){};
		\end{tikzpicture}
		\caption{Beispiel für einen leeren Platz und einen Platz mit einer (abstrakten) Marke.}
		\label{fig:platz}
	\end{figure}

	Die Objekte, die von Plätzen gelagert werden, werden allgemein als \textbf{Marken} bezeichnet. Grundsätzlich kann ein Platz beliebig viele Marken enthalten. Außerdem können in einem Modell unterschiedliche Typen von Marken verwendet werden. Eine Marke kann ein konkretes Objekt aus der realen Welt repräsentieren (Maschinen, Werkstücke, Menschen, ...). Es können jedoch auch abstrakte Marken verwendet werden, die dann meist als schwarzer Punkt dargestellt werden, wie in Abbildung~\ref{fig:platz} zu sehen. Abstrakte Marken sind sehr nützlich, um Zustände, in dem sich das System befinden kann, zu modellieren.

	\paragraph{Transitionen} sind die aktiven Komponenten eines Petrinetzes.
    Sie beschreiben eine elementaren Aktion, die Dinge erzeugen, transportieren, verändern oder vernichten \footnote{\cite{hu_berlin:petrinetze:plaetze}}.
	Eine Transition wird durch ein Quadrat dargestellt:
	\begin{figure}[h]
		\centering
		\begin{tikzpicture}
			\node[transition]{};
		\end{tikzpicture}
		\caption{Beispiel für eine Transition}
		\label{fig:transition}
	\end{figure}

	Transitionen können zusätzlich noch mit einer Formel beschriftet werden. Diese ist eine Vorbedingung, die erfüllt sein muss, damit die Transition aktiviert werden kann. Wann genau eine Transition aktiviert ist und was das bedeutet, wird später in Kapitel~\ref{ssub:ablaufregeln} noch diskutiert.

	\paragraph{Kanten} sind selbst keine Komponenten, sondern stellen Beziehungen zwischen Komponenten her. Kanten werden durch einen Pfeil gekennzeichnet und verbinden immer eine Transition mit einem Platz oder einen Platz mit einer Transition. Niemals jedoch verbinden sie zwei Plätze oder zwei Transitionen.

	Kanten, die von einer Transition auf eine Stelle zeigen, drücken aus, dass diese Transition Objekte erzeugt und sie in der Stelle ablegt. Kanten, die von einer Stelle auf eine Transition zeigen, drücken aus, dass diese Transition Objekte aus der Stelle entnimmt.

	Eine Kante kann zusätzlich noch mit einem Term beschriftet werden. Das Ergebnis dieses Terms müssen Marken sein. Damit wird angegeben, wie viele und welche Marken die Transition aus der Stelle entnimmt beziehungsweise in ihr erzeugt. Im Normalfall ist die Beschriftung einfach ein konstanter Term, es sind jedoch auch Terme mit beliebigen Variablen erlaubt.

	\begin{figure}[ht]
		\centering
		\begin{tikzpicture}
			\node[place, tokens=1](konsStart){};
			\node[transition, right=3cm of konsStart](konsEnd){}
				edge[pre] node[token,below=1pt,label=above:konsumiert]{} (konsStart);

			\node[transition, below= of konsStart](prodStart){};
			\node[place, tokens=1, right=3cm of prodStart](prodEnd){}
				edge [pre] node[token,below=1pt,label=above:produziert]{} (prodStart);
		\end{tikzpicture}
		\caption{Beispiele für Kanten}
		\label{fig:kanten}
	\end{figure}

\section{Ablaufregeln}
\label{ssub:ablaufregeln}
	Der allgemeine Aufbau von Petrinetzen ist nun bekannt. Nach welchen Regeln sich ein Petrinetz genau verhält, wurde jedoch noch nicht festgelegt. In den folgenden Abschnitten sollen diese Regeln erläutert werden.
	\\\\
	Als \textbf{Markierung} eines Petrinetzes bezeichnet man die Verteilung von Marken auf die Plätze. In Abbildung~\ref{fig:markierung1}~und~\ref{fig:markierung2} sind zwei unterschiedliche Markierungen eines Petrinetzes dargestellt.
	\begin{figure}[t]
		\centering
		\begin{tikzpicture}
			\node[transition,label=left:Produkt herstellen](t){};

			\node[place, below left=of t, label=below:Auftrag liegt vor, tokens=2](A){};
			\node[place, above left=of t, label=Lagerbestand](M){}
				[children are tokens]
				child{node[draw,regular polygon,regular polygon sides=3,inner sep=1pt]{}}
				child{node[draw,diamond,inner sep=1.4pt]{}}
				child{node[draw,diamond,inner sep=1.4pt]{}};
			\node[place, right=of t, label=Endprodukt](P){};

			\draw (t)	edge[pre]	node[token,left=2pt]{}	(A)
						edge[pre]	node[draw,regular polygon,regular polygon sides=3,inner sep=1pt,left=3pt]{}	(M)
						edge[post]	node[draw, below=1pt]{}	(P);
		\end{tikzpicture}
		\caption{Markierung 1 eines Petrinetzes}
		\label{fig:markierung1}
	\end{figure}
	\begin{figure}[t]
		\centering
		\begin{tikzpicture}
			\node[transition,label=left:Produkt herstellen](t){};

			\node[place, below left=of t, label=below:Auftrag liegt vor,tokens=1](A){};
			\node[place, above left=of t, label=Lagerbestand](M){}
				[children are tokens]
				child{node[draw,diamond,inner sep=1.4pt]{}}
				child{node[draw,diamond,inner sep=1.4pt]{}};

			\node[place, right=of t, label=Endprodukt](P){}
				[children are tokens]
				child{node[draw]{}};

			\draw (t)	edge[pre]	node[token,left=2pt]{}	(A)
						edge[pre]	node[draw,regular polygon,regular polygon sides=3,inner sep=1pt,left=3pt]{}	(M)
						edge[post]	node[draw, below=1pt]{}	(P);
		\end{tikzpicture}
		\caption{Markierung 2 eines Petrinetzes}
		\label{fig:markierung2}
	\end{figure}
	\\\\
	Als nächstes wird erklärt, wann man eine Transition als \textbf{aktiviert} bezeichnet. Anschaulich gesehen ist eine Transition $t$ in einer bestimmten Markierung aktiviert, wenn alle Plätze, aus denen $t$ Marken entnimmt, dafür genug Marken beinhalten.
	Formal muss also jeder Platz, von dem eine Kante auf $t$ zeigt, von jedem Markentyp genausoviele oder mehr Marken beinhalten als in der Beschriftung der Kante angegeben.

	Im Petrinetz aus Abbildung~\ref{fig:markierung1} entnimmt die Transition $Produkt herstellen$ eine dreieckige Marke aus dem Platz $Lagerbestand$ und eine schwarze Marke aus $Auftrag liegt vor$. Beide Plätze enthalten die geforderten Marken. Also ist die Transition aktiviert.

	In Abbildung~\ref{fig:markierung2} ist das selbe Petrinetz in einer anderen Markierung dargestellt. Hier enthält $Auftrag liegt vor$ genug schwarze Marken. Aus dem Platz $Lagerbestand$ entnimmt die Transition jedoch eine dreieckige Marke, aber in dem Platz sind nur Rautenfürmige Marken enthalten. Daher ist die Transition in Markierung 2 nicht aktiviert.

	Diese Vorgehensweise funktioniert für Kanten, die mit konstanten Termen beschriftet sind. Wenn im Term einer Kante Variablen vorkommen, muss zunächst eine Belegung der Variablen mit echten Werten gefunden werden, für die diese Bedingung zutrifft. Außerdem muss die Formel in der Transition für die Variablenbelegung wahr sein. Wenn eine passende Variablenbelegung existiert, dann ist die Transition in der entsprechenden Markierung unter dieser Variablenbelegung aktiviert. Andernfalls ist die Transition nicht aktiviert.

	\begin{figure}[h]
		\centering
		\begin{tikzpicture}
			\node[place,label=below:Platz 1](p1){}
				[children are tokens]
				child{node[token]{}}
				child{node[draw,fill]{}};
			\node[place,label=below:Platz 2,below=of p1](p2){}
				[children are tokens]
				child{node[token]{}}
				child{node[draw,fill]{}};
			\node[place,label=below:Platz 3,below=of p2,tokens=1](p3){};

			\node[transition,right=3cm of p2](t){}
				edge[pre,bend right]	node[right]{X}		(p1)
				edge[pre,bend right]	node[below]{X}		(p2)
				edge[post,bend left]	node[below]{X,Y}	(p2)
				edge[pre,bend left]		node[below]{Y}		(p3);

			\node[place,right=of t,label=below:Platz 4](pr){}
				edge[pre] node[below]{X} (t);
		\end{tikzpicture}
		\caption{Ein Petrinetz mit Variablen in den Kantenbeschriftungen}
		\label{fig:varkanten1}
	\end{figure}

	Abbildung~\ref{fig:varkanten1} zeigt ein Petrinetz mit Variablen in den Kantenbeschriftungen. In der gezeigten Markierung gibt es zwei Variablenbelegungen, für die die Transition aktiv ist:
	\begin{itemize}
		\item $X$ = \tikz{\node[fill]{}}	\hspace{1cm} $Y$ = \tikz{\node[token]{}} 
		\item $X$ = \tikz{\node[token]{}}	\hspace{1cm} $Y$ = \tikz{\node[token]{}}
	\end{itemize}
	Eine Variablenbelegung mit $Y$ = \tikz{\node[fill]{}}, bei der die Transition aktiviert ist, kann es in der gezeigten Markierung nicht geben, denn in $Platz 3$ ist ausschließlich eine kreisförmige Marke vorhanden.
	
	Wenn eine Transition aktiviert ist, dann kann sie \textbf{eintreten}. Beim Ablauf eines Petrinetzes kann in einer Markierung immer nur eine Transition eintreten. Welche der aktivierten Transitionen eintritt, ist nicht deterministisch bestimmt. Durch das Eintreten einer Transition wird eine neue Markierung für das Petrinetz erzeugt. Dabei werden aus allen Plätzen, von denen eine Kante auf die Transition zeigen, die in der Kantenbeschriftung angegebenen Marken entfernt. Außerdem werden zu allen Plätzen, auf die eine Kante zeigt, die bei der Transition beginnt, die in der Kantenbeschriftung angegebenen Marken hinzugefügt.

	Wenn die Kantenbeschriftungen Variablen enthalten, müssen zunächst die Werte aus der Variablenbelegung, für die die Transition aktiv ist, eingesetzt werden. 

	Wenn die Transition $Produkt herstellen$ in Abbildung~\ref{fig:markierung1} eintritt, dann wird aus de Platz $Lagerbestand$ ein Marke \tikz{\node[draw,regular polygon,regular polygon sides=3,inner sep=1pt]{}} entfernt und aus dem Platz $Auftrag liegt vor$ wird eine Marke \tikz{\node[token]{}} entfernt. Außerdem wird zum Platz $Endprodukt$ eine Marke \tikz{\node[draw]{}} hinzugefügt. So entsteht die Markierung, die in Abbildung~\ref{fig:markierung2} zu sehen ist.

	Wie oben bereits erläutert, ist die Transition aus Abbildung~\ref{fig:varkanten1} für die Variablenbelegung $X$ = \tikz{\node[fill]{}}, $Y$ = \tikz{\node[token]{}} aktiviert. Wenn die Transition unter dieser Variablenbelegung eintritt, wird aus $Platz 3$ eine Marke \tikz{\node[token]{}} und aus $Platz 1$ eine Marke \tikz{\node[fill]{}} entfernt. Aus $Platz 2$ wird zunächst ebenfalls eine Marke \tikz{\node[fill]{}} entfernt und die Marken \tikz{\node[fill]{}} und \tikz{\node[token]{}} wieder hinzugefügt. Außerdem wird zu $Platz 4$ eine Marke \tikz{\node[fill]{}} hinzugefügt.

	\begin{figure}[h]
		\centering
		\begin{tikzpicture}
			\node[place,label=below:Platz 1,tokens=1](p1){};
			\node[place,label=below:Platz 2,below=of p1](p2){}
				[children are tokens]
				child{node[token]{}}
				child{node[token]{}}
				child{node[draw,fill]{}};
			\node[place,label=below:Platz 3,below=of p2](p3){};

			\node[transition,right=3cm of p2](t){}
				edge[pre,bend right]	node[right]{X}		(p1)
				edge[pre,bend right]	node[below]{X}		(p2)
				edge[post,bend left]	node[below]{X,Y}	(p2)
				edge[pre,bend left]		node[below]{Y}		(p3);

			\node[place,right=of t,label=below:Platz 4](pr){}
				[children are tokens]
				child{node[draw,fill]{}};
				
			\draw (pr) edge[pre] node[below]{X} (t);
		\end{tikzpicture}
		\caption{Das Petrinetz aus Abbildung~\ref{fig:varkanten1} nach dem Eintreten der Transition.}
		\label{fig:varkanten2}
	\end{figure}












































% @:    so sagen?
\paragraph{Schnittstellen} ermöglichen auf Aktivitäten aus der Umgebung einzugehen.
Mit Hilfe der bisherig vorgestellten Bestandteile lassen sich bereits geschlossene Systeme modellieren.
Sollen allerdings Interaktionen eines Geschäftsprozesses mit z.B. einem Kunden modelliert werden, benötigt man Schnittstellen.
Diese werden durch eine Transition dargestellt.

\paragraph{Kalte Transitionen} dienen der Modellierung von Transitionen, bei denen nicht klar ist ob sie eintreffen.
Wird z.B. ein Prozess von einem Kunden durch dessen Bezahlung ausgelöst, ist im Vorhinein nicht klar ob dieses Ereignis jemals eintreffen wird.
Ist das Gegenteil der Fall wird die Transition als \enquote{warm} bezeichnet.
Kalte Transitionen werden durch eine Transition dargestellt, die ein $\epsilon$ beinhalten dargestellt.
\begin{center}
	\begin{tikzpicture}
		\node[transition]{$\epsilon$};
	\end{tikzpicture}
\end{center}

\paragraph{Zähler} erleichtern insbesondere die Modellierung von Lagern.
Können z.B. in einem Lager maximal 10 Stücke eines Materials gespeichert werden.
Um dies anschaulich zu modellieren wird der Transaktion ein Parameter $x$ übergeben.
Dabei wird die Transition als Funktion aufgefasst und mit Rückgabewert $x - 1$.
\begin{center}
	\begin{tikzpicture}
		\node(left){};
		\node[right=3cm of left](right){};
		\node[transition, right=1cm of left, left=1cm of right](transition){$x>0$}
		edge[pre, dashed] (left)
		edge[post, dashed] (right);
		\node[place, below= of transition] {10}
		edge[pre, anchor=east, transform canvas={xshift=-1mm}] node {$x$} (transition)
		edge[post, anchor=west, transform canvas={xshift=1mm}] node {$x-1$} (transition);

	\end{tikzpicture}
\end{center}
Somit lässt sich verhindern, dass die Transition mehr als 10 mal ausgeführt wird.



\section{Variationen}
Mit den zunehmenden Anforderungen an Petrinetze sind zahlreiche Variationen von Petrinetzen entstanden.
Dabei werden die Varianten in zwei Kategorien gegliedert:
\begin{itemize}
    \item low-level Petrinetze
    \item high-level Petrinetze
\end{itemize}

\subsection{Elementare Petrinetze}
Häufig werden Petri Netze dazu verwendet darzustellen, wo ein Kontrollfluss aktuell steht. \footnote{(vgl. \cite[Seite 39]{hu_berlin:petrinetze})}
Will man einen Prozess genauer analysieren geht dies am einfachsten in dem man den Prozess abstrakt darstellt.
Dafür bieten sich elementare Petrinetze an, die auch als elementare Systemnetze bezeichnet werden.\\
Bei einem elementaren Petrinetz werden die Marken auf den Kanten weggelassen. Außerdem gibt es nur eine Form von Marken, die in der Regel durch einen schwarzen Punkt \textbullet visualisiert wird.
Dabei beschränkt sich ein elementares Petrinetz auf Transitionen, Plätze und Marken.
Aufgrund der abstrakten Modellierung und geringen Anzahl an Komponenten gehören elementare Petrinetze zu den low-level Petrinetzen.

\subsection{Gefärbte Petrinetze}
Gefärbte Petrinetze werden hauptsächlich zur Modellierung komplexer Systeme verwenden, da Sie zu der Gruppe der high-level Petrinetzen gehören.
Im Unterschied zu anderen Petrinetzen sind die gefärbten Petrinetze wesentlich übersichtlicher, können allerdings jeder Zeit wieder in ein low-level Petrinetz zurückgeführt werden.
Dabei bieten die gefärbten Petrinetze weitere Funktionalitäten wie z.B.:
\begin{itemize}
    \item Typen
    \item Knotenfunktionen
    \item Typfunktionen
    \item Wächterfunktionen
    \item Kantenausdrucksfunktionen
    \item Initialisierungsfunktionen
\end{itemize}
\footnote{(vgl.: \cite{tu_dresden:petrinetze})}
auf die in diesem Dokument nicht weiter eingegangen wird.

%TODO

\subsection{Algorithmen}
%TODO

\subsubsection{Verifikation}
%TODO

\subsubsection{Analyse}
%TODO

