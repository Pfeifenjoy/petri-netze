\section{Historie}
Carl Adam Petri, geboren am 12. Juli 1926 in Leibzig und gestorben am 2. Juli 2010, ist der Schöpfer der Petrinetze. \\
Bereits im Alter von 13 Jahren versuchte er sein Wissen in chemischen Prozessen zu visualisiern und diese Prozesse abzubilden.\\ 
Bereits im Jahre 1941 begann er mit seinen theoretischen Forschungen, die damals durch den Entwickler, der ersten frei programmierbaren Computer,Konrad Zuse(1910-95), stark beeinflusst worden sind.\\
Zu seiner Studienzeit und der daraufhin folgenden Zeit als Assisstent, an der technischen Universität Hannover(1950-56) und Bonn(1956-62), hat er gemerkt, dass es nötig ist einen theoretischen Framework zu schaffen, um die fundamentalen Probleme zu lösen, welche durch die Struktur und Verwendung der Computer-Technologie entstanden sind. \\
Dieser Framework sollte unabhängig von der aktuellen technischen Wissenschaft sein, jedoch mit den physikalischen Gesetzten übereinstimmen.
Aus diesem Grund lehnte er Modelle über zentrale Kontrollen ab. Daher entwickelte er ein Modell mit einer dezentralisierten Kontrolle.
Diese Ideen beschrieb er in seiner Disertation über "Kommunikation mit Automaten".
Auch wenn Petri nicht sehr viel von Petrinetzen in dieser Disertation erwähnt hatte, so hat er jedoch deren Features in mathematische Notatioenn integriert.
Petrinetze, so wie wir sie heute kennen, hat Petri zum erstenmal bei einer Konferenz im Jahre 1965 erwähnt. \\ Jedoch sind seine Theorien zu dieser Zeit zu weit voraus gewesen, weshalb er in der wissenschaftlichen Community keinen Anklang gefunden hat. Dies wandelte sich mit der entwicklung parallelisierter Computer-Systeme und dezentralisierten Netzwerken. Im Jahre 2008 erhielt Petri den "`Computer Pioneer Award"'.\footnote{Vgl. http://www.informatik.uni-hamburg.de/TGI/mitarbeiter/profs/petri/obituary.html} \\ \\ 
