\section{Historie}
Carl Adam Petri, geboren am 12. Juli 1926 in Leipzig und gestorben am 2. Juli 2010, ist der Schöpfer der Petrinetze. \\
Bereits im Alter von 13 Jahren versuchte er sein Wissen über chemischen Prozesse zu visualisieren und diese Prozesse abzubilden.\\
Im Jahre 1941 begann er mit seinen theoretischen Forschungen, die damals durch den Entwickler, der ersten frei programmierbaren Computer, Konrad Zuse (1910-95), stark beeinflusst worden sind.\\
Zu seiner Studienzeit und der daraufhin folgenden Zeit als Assistent an der technischen Universität Hannover (1950-56) und Bonn (1956-62) merkte er, dass es nötig ist einen Formalismus zur Lösung elementarer Probleme der Informatik zu schaffen.
%Der Formalismus sollte unabhängig von der aktuellen technischen Wissenschaft sein, jedoch mit den physikalischen Gesetzten übereinstimmen.
Dabei wollte er eine Alternative zu Automaten schaffen.
Aus diesem Grund lehnte er Modelle mit einem globalen Zustand ab und entwickelte ein aus dezentralisierten Zuständen bestehendes Modell.
Diese Ideen beschrieb er in seiner Dissertation über \enquote{Kommunikation mit Automaten}.
Auch wenn Petri nicht sehr viel von Petrinetzen in dieser Dissertation erwähnt hatte, so hat er jedoch deren Eigenschaften in mathematische Notationen integriert.
Petrinetze, so wie wir sie heute kennen, hat Petri zum ersten Mal bei einer Konferenz im Jahre 1965 erwähnt. \\ Jedoch gab es für seine Theorien zu seiner Zeit wenig Anwendungen, weshalb sie in der Wissenschaftsgemeinde keinen Anklang fanden. Dies wandelte sich mit der Entwicklung parallelisierter Computer-Systeme und dezentralisierter Netzwerke. Im Jahre 2008 erhielt Petri den \enquote{Computer Pioneer Award}.\footnote{Vgl. http://www.informatik.uni-hamburg.de/TGI/mitarbeiter/profs/petri/obituary.html}
