%!TEX root = ../dokumentation.tex

\chapter{Einleitung}
Der Formalismus der Petri-Netze ging aus der Theorethischen Informatik hervor
und diente ursprünglich lediglich der Beschreibung von vernetzten technischen Systemen.

Meist werden technische Systeme mit Hilfe eines Automaten beschrieben.
Mit der ansteigenden Komplexität großer System wächst die Anzahl der Zustände potentiell exponentiell an.
Dies wird besonders bei der Modellierung von vernetzen Systemen schnell unübersichtlich.
Um trotzdem solche Systeme modellieren zu können, werden sie mit Hilfe von Petri-Netzen beschrieben.
Der Formalismus der Petri-Netze beschränkt sich heutzutage nicht mehr nur auf die Themenbereiche der Informatik,
sondern hat sich auch als bewährtes Mittel bei der Modellierung von Geschäftsprozessen etabliert.
Die große Stärke der Petri-Netze besteht darin nebenläufige Ereignisse darzustellen,
wodurch so genannte \enquote{deadlocks} \footnote{\url{https://technet.microsoft.com/en-us/library/ms177433(v=sql.105).aspx} [Stand: Juni 2015]}
in Geschäftsprozessen frühzeitig erkannt werden können und
die Laufzeit eines Systems durch Analyse verkürzt werden kann.
Außerdem lassen sich komplexe Systeme durch eine Modellierung besser verstehen,
wodurch z.B. ein Auftraggeber ohne allzu große Details über die Implementierung eines Prozesses
diesen nachvollziehen kann.
Im Verlaufe dieses Dokumentes soll hauptsächlich auf die Modellierung von Geschäftsprozess eingegangen werden.

Das Dokument erklärt was Petri-Netze im allgemeinen sind und
welche Variationen sich im Verlaufe der Geschichte entwickelt haben.
Außerdem werden Algorithmen beschrieben, die sich auf Petri-Netze anwenden lassen.
Die Theorie und insbesondere die damit verbundenen Algorithmen
werden anhand der Planung einer Produktionsanlage für einen Baukastens
mit Hilfe von Petri-Netzen modellhaft angewandt.

%TODO Sollte noch etwas länger werden

\section{Entwicklung}
%TODO
