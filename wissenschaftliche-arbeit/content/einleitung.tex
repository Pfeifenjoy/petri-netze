%!TEX root = ../dokumentation.tex

\chapter{Einleitung}
\section{Allgemeines}
Der Formalismus der Petri-Netze ging aus der Theorethischen Informatik hervor
und diente ursprünglich lediglich der Beschreibung von vernetzten technischen Systemen.

Meist wurden technische Systeme mit Hilfe eines Automaten beschrieben.
Mit der ansteigenden Komplexität großer Systeme wächst die Anzahl der Zustände potentiell exponentiell an.
Dies wird besonders bei der Modellierung von vernetzen Systemen schnell unübersichtlich.
Um trotzdem solche Systeme modellieren zu können, werden sie mit Hilfe von Petri-Netzen beschrieben.
Der Formalismus der Petri-Netze beschränkt sich heutzutage nicht mehr nur auf die Themenbereiche der Informatik,
sondern hat sich auch als bewährtes Mittel bei der Modellierung von Geschäftsprozessen etabliert.
Die große Stärke der Petri-Netze besteht darin nebenläufige Ereignisse darzustellen,
wodurch so genannte \enquote{deadlocks} \footnote{(vgl. \cite{microsoft:deadlocks})}
in Geschäftsprozessen frühzeitig erkannt werden können und
die Laufzeit eines Systems durch Analyse verkürzt werden kann.
Außerdem lassen sich komplexe Systeme durch eine Modellierung besser verstehen,
wodurch z.B. ein Auftraggeber ohne allzu große Details über die Implementierung eines Prozesses
diesen nachvollziehen kann.
Im Verlaufe dieses Dokumentes soll hauptsächlich auf die Modellierung von Geschäftsprozessen eingegangen werden.

Das Dokument erklärt was Petri-Netze im allgemeinen sind und
welche Variationen sich im Verlaufe der Geschichte entwickelt haben.
Außerdem werden Algorithmen beschrieben, die sich auf Petri-Netze anwenden lassen.
Die Theorie und insbesondere die damit verbundenen Algorithmen
werden anhand der Planung einer Produktionsanlage für einen Baukasten
mit Hilfe von Petri-Netzen modellhaft angewandt.
\newpage
\section{Historie}
Carl Adam Petri, geboren am 12. Juli 1926 in Leipzig und gestorben am 2. Juli 2010, ist der Schöpfer der Petrinetze. \\
Bereits im Alter von 13 Jahren versuchte er sein Wissen über chemischen Prozesse zu visualisieren und diese Prozesse abzubilden.\\
Im Jahre 1941 begann er mit seinen theoretischen Forschungen, die damals durch den Entwickler, der ersten frei programmierbaren Computer, Konrad Zuse (1910-95), stark beeinflusst worden sind.\\
Zu seiner Studienzeit und der daraufhin folgenden Zeit als Assistent an der technischen Universität Hannover (1950-56) und Bonn (1956-62) merkte er, dass es nötig ist einen Formalismus zur Lösung elementarer Probleme der Informatik zu schaffen.
%Der Formalismus sollte unabhängig von der aktuellen technischen Wissenschaft sein, jedoch mit den physikalischen Gesetzten übereinstimmen.
Dabei wollte er eine Alternative zu Automaten schaffen.
Aus diesem Grund lehnte er Modelle mit einem globalen Zustand ab und entwickelte ein aus dezentralisierten Zuständen bestehendes Modell.
Diese Ideen beschrieb er in seiner Dissertation über \enquote{Kommunikation mit Automaten}.
Auch wenn Petri nicht sehr viel von Petrinetzen in dieser Dissertation erwähnt hatte, so hat er jedoch deren Eigenschaften in mathematische Notationen integriert.
Petrinetze, so wie wir sie heute kennen, hat Petri zum ersten Mal bei einer Konferenz im Jahre 1965 erwähnt. \\ Jedoch gab es für seine Theorien zu seiner Zeit wenig Anwendungen, weshalb sie in der Wissenschaftsgemeinde keinen Anklang fanden. Dies wandelte sich mit der Entwicklung parallelisierter Computer-Systeme und dezentralisierter Netzwerke. Im Jahre 2008 erhielt Petri den \enquote{Computer Pioneer Award}.\footnote{(Vgl. \cite{uni_hamburg:pioneerAward})}

