%!TEX root = petri-netze.tex
\section{Theorie}

%TODO Hier fehlt noch ein einführender Text zwischen der Über- und der Unterüberschrift

\subsection{Aufbau von Petrinetzen} % (fold)
\label{sub:aufbau_von_petrinetzen}
	Im Prinzip können Petri-Netze als Graphen aufgefasst werden.
	Diese Graphen besitzen allerdings besondere Knoten, die als Komponenten bezeichnet werden.
	Im folgenden werden die Komponenten eines Petri-Netzes beschrieben.

	\paragraph{Plätze} sind die passiven Komponenten eines Petri-Netzes. Sie dienen repräsentieren Zustände oder lagern Objekte.
	Ein Platz wird durch einen Kreis dargestellt:
	\begin{figure}[h]
		\centering
		\begin{tikzpicture}
			\node[place](emptyPlace){};
			\node[below=0.2cm of emptyPlace](emptyPlaceLabel){Leerer Platz};
			\node[right= of emptyPlaceLabel](filledPlaceLabel){Gefüllter Platz};
			\node[place, tokens=1, above=0.2cm of filledPlaceLabel](filledPlace){};
		\end{tikzpicture}
		\caption{Beispiel für einen leeren Platz und einen Platz mit einer (abstrakten) Marke.}
		\label{fig:platz}
	\end{figure}

	Die Objekte, die von Plätzen gelagert werden, werden allgemein als \textbf{Marken} bezeichnet. Grundsätzlich kann ein Platz beliebig viele Marken enthalten. Eine Marke kann ein konkretes Objekt aus der realen Welt repräsentieren (Maschinen, Werkstücke, Menschen, ...). Es können jedoch auch abstrakte Marken verwendet werden, die dann meist als schwarzer Punkt dargestellt werden, wie in Abbildung~\ref{fig:platz} zu sehen. Abstrakte Marken sind sehr nützlich, um Zustände, in dem sich das modellierte System befinden kann, zu modellieren.

	\paragraph{Transitionen} sind die aktiven Komponenten eines Petri-Netzes.
	Sie beschreiben eine elementaren Aktion, die Dinge erzeugen, transportieren, verändern oder vernichten \footnote{\label{hu_berlin}\url{https://www2.informatik.hu-berlin.de/top/lehre/WS05-06/se_systementwurf/Petrinetze-1.pdf}[Stand: Juni 2015]}.
	Eine Transition wird durch ein Quadrat dargestellt:
	\begin{figure}[h]
		\centering
		\begin{tikzpicture}
			\node[transition]{};
		\end{tikzpicture}
		\caption{Beispiel für eine Transition}
		\label{fig:transition}
	\end{figure}

	Transitionen können zusätzlich noch mit formeln beschriftet werden. Zur Bedeutung dieser Formeln kommen wir später.

	\paragraph{Kanten} sind selbst keine Komponenten, sondern stellen Beziehungen zwischen Komponenten her. Kanten werden durch einen Pfeil gekennzeichnet und verbinden immer eine Transition mit einem Platz oder einen Platz mit einer Transition. Niemals jedoch verbinden sie zwei Plätze oder zwei Transitionen.

	Kanten, die von einer Transition auf eine Stelle zeigen, drücken aus, dass diese Transition Objekte erzeugt und sie in der Stelle ablegt. Kanten, die von einer Stelle auf eine Transition zeigen, drücken aus, dass diese Transition Objekte aus der Stelle entnimmt.

	\begin{figure}[h]
		\centering
		\begin{tikzpicture}
			\node[place, tokens=1](konsStart){};
			\node[transition, right=3cm of konsStart](konsEnd){}
				edge[pre] node[token,below=1pt,label=above:konsumiert]{} (konsStart);
			
			\node[transition, below= of konsStart](prodStart){};
			\node[place, tokens=1, right=3cm of prodStart](prodEnd){}
				edge [pre] node[token,below=1pt,label=above:produziert]{} (prodStart);
		\end{tikzpicture}
		\caption{Beispiele für Kanten}
		\label{fig:kanten}
	\end{figure}

\subsection{Ablaufregeln}
\label{ssub:ablaufregeln}
	Der allgemeine Aufbau von Petrinetzen ist nun bekannt. Nach welchen Regeln sich ein Petrinetz genau verhält, wurde jedoch noch nicht festgelegt. In den folgenden Abschnitten sollen diese Regeln erläutert werden.

	\paragraph{Als Markierung} eines Petrinetzes bezeichnet man die Verteilung von Marken auf die Plätze. In Abbildung~\ref{fig:markierungen} sind zwei verschiedene Markierungen eines Petrinetzes dargestellt.
	\begin{figure}[ht]
		\centering
		\begin{tikzpicture}
			\node[transition,label=left:Produkt herstellen](t){};

			\node[place, below left=of t, label=below:Auftrag liegt vor, tokens=1](A){};
			\node[place, above left=of t, label=Lagerbestand, tokens=2](M){};
			\node[place, right=of t, label=Endprodukt](P){};

			\draw (t)	edge[pre]	node[token,left=2pt]{}	(A)
						edge[pre]	node[token,left=2pt]{}	(M)
						edge[post]	node[draw, below=1pt]{}	(P);
		\end{tikzpicture}

		\begin{tikzpicture}
			\node[transition,label=left:Produkt herstellen](t){};

			\node[place, below left=of t, label=below:Auftrag liegt vor](A){};
			\node[place, above left=of t, label=Lagerbestand, tokens=1](M){};
			\node[place, right=of t, label=Endprodukt](P){}
				[children are tokens]
				child{node[draw]{}};

			\draw (t)	edge[pre]	node[token,left=2pt]{}	(A)
						edge[pre]	node[token,left=2pt]{}	(M)
						edge[post]	node[draw, below=1pt]{}	(P);
		\end{tikzpicture}
		\caption{Beispiel für zwei verschiedene Markierungen eines Petrinetzes}
		\label{fig:markierungen}
	\end{figure}


















































%TODO @Tobias: Kann man das so sagen?
\paragraph{Schnittstellen} ermöglichen auf Aktivitäten aus der Umgebung einzugehen.
Mit Hilfe der bisherig vorgestellten Bestandteile lassen sich bereits geschlossene Systeme modellieren.
Sollen allerdings Interaktionen eines Geschäftsprozesses mit z.B. einem Kunden modelliert werden, benötigt man Schnittstellen.
Diese werden durch eine Transition dargestellt.

\paragraph{Kalte Transitionen} dienen der Modellierung von Transitionen, bei denen nicht klar ist ob sie eintreffen.
Wird z.B. ein Prozess von einem Kunden durch dessen Bezahlung ausgelöst, ist im Vorhinein nicht klar ob dieses Ereignis jemals eintreffen wird.
Ist das Gegenteil der Fall wird die Transition als \enquote{warm} bezeichnet.
Kalte Transitionen werden durch eine Transition dargestellt, die ein $\epsilon$ beinhalten dargestellt.
\begin{center}
	\begin{tikzpicture}
		\node[transition]{$\epsilon$};
	\end{tikzpicture}
\end{center}

\paragraph{Zähler} erleichtern insbesondere die Modellierung von Lagern.
Können z.B. in einem Lager maximal 10 Stücke eines Materials gespeichert werden.
Um dies anschaulich zu modellieren wird der Transaktion ein Parameter $x$ übergeben.
Dabei wird die Transition als Funktion aufgefasst und mit Rückgabewert $x - 1$.
\begin{center}
	\begin{tikzpicture}
		\node(left){};
		\node[right=3cm of left](right){};
		\node[transition, right=1cm of left, left=1cm of right](transition){$x>0$}
		edge[pre, dashed] (left)
		edge[post, dashed] (right);
		\node[place, below= of transition] {10}
		edge[pre, anchor=east, transform canvas={xshift=-1mm}] node {$x$} (transition)
		edge[post, anchor=west, transform canvas={xshift=1mm}] node {$x-1$} (transition);

	\end{tikzpicture}
\end{center}
Somit lässt sich verhindern, dass die Transition mehr als 10 mal ausgeführt wird.



\subsection{Variationen}
%TODO

\subsection{Algorithmen}
%TODO

\subsubsection{Verifikation}
%TODO

\subsubsection{Analyse}
%TODO

