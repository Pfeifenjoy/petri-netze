\section{Theorie}
%TODO @Tobias bitte überprüfe nochmal Begrifflichkeiten
Im Prinzip können Petri-Netze als Graphen aufgefasst werden.
Diese Graphen besitzen allerdings besondere Knoten, die als Komponenten bezeichnet werden.
Im folgenden sollen die Komponenten eines Petri-Netzes beschrieben werden.

\paragraph{Transitionen} sind Komponenten eines Petri-Netzes.
Sie beschreiben eine elementaren Aktion, die Dinge erzeugen, transportieren, verändern oder vernichten \footnote{\label{hu_berlin}\url{https://www2.informatik.hu-berlin.de/top/lehre/WS05-06/se_systementwurf/Petrinetze-1.pdf}[Stand: Juni 2015]} kann.
Eine Transition wird durch ein Quadrat dargestellt:
\begin{center}
    \begin{tikzpicture}
        \node[transition]{};
    \end{tikzpicture}
\end{center}

\paragraph{Plätze} sind Komponenten eines Petri-Netzes an denen Objekte, wie z.B. Werkzeuge, Materialien, Produkte oder auch Rezepte gelagert werden können.
Sie werden zum lagern, speichern, sichtbar machen oder Zustände repräsentieren \footref{hu_berlin} verwendet.
Ein Platz wird durch einen Kreis dargestellt:
\begin{center}
    \begin{tikzpicture}
        \node[place](emptyPlace){};
        \node[below=0.2cm of emptyPlace](emptyPlaceLabel){Leerer Platz};
        \node[right= of emptyPlaceLabel](filledPlaceLabel){Gefüllter Platz};
        \node[place, tokens=1, above=0.2cm of filledPlaceLabel](filledPlace){};
    \end{tikzpicture}
\end{center}

\paragraph{Kanten} sind selbst keine Komponenten, sondern stellen Beziehungen zwischen Komponenten her.
Sie bilden logische Zusammenhänge, Zugriffsrechte und eine Kausale Ordnung \footref{hu_berlin}.
Sie werden durch einen Pfeil gekennzeichnet und Stellen folgende Beziehungen da:
\begin{center}
    \begin{tikzpicture}
        \node[place, tokens=1](konsStart){};
        \node[transition, right=3cm of konsStart](konsEnd){}
        edge [pre] node[anchor=south] {konsumiert}(konsStart);
        \node[transition, below= of konsStart](prodStart){};
        \node[place, tokens=1, right=3cm of prodStart](prodEnd){}
        edge [pre] node[anchor=south] {produziert}(prodStart);
    \end{tikzpicture}
\end{center}
Dabei sind folgende Relation unzulässig:
\begin{center}
    \begin{tikzpicture}
        \node[transition](konsStart){};
        \node[transition, right=3cm of konsStart](konsEnd){};
        \node[place, below= of konsStart](prodStart){};
        \node[place, right=3cm of prodStart](prodEnd){};
    \end{tikzpicture}
\end{center}

%TODO @Tobias: Kann man das so sagen?
\paragraph{Schnittstellen} ermöglichen auf Aktivitäten aus der Umgebung einzugehen.
Mit Hilfe der bisherig vorgestellten Bestandteile lassen sich bereits geschlossene Systeme modellieren.
Sollen allerdings Interaktionen eines Geschäftsprozesses mit z.B. einem Kunden modelliert werden, benötigt man Schnittstellen.
Diese werden durch eine Transition dargestellt.

\paragraph{Kalte Transitionen} dienen der Modellierung von Transitionen, bei denen nicht klar ist ob sie eintreffen.
Wird z.B. ein Prozess von einem Kunden durch dessen Bezahlung ausgelöst, ist im Vorhinein nicht klar ob dieses Ereignis jemals eintreffen wird.
Ist das Gegenteil der Fall wird die Transition als \enquote{warm} bezeichnet.
Kalte Transitionen werden durch eine Transition dargestellt, die ein $\epsilon$ beinhalten dargestellt.
\begin{center}
    \begin{tikzpicture}
        \node[transition]{$\epsilon$};
    \end{tikzpicture}
\end{center}

\paragraph{Zähler}

\subsection{Variationen}
%TODO

\subsection{Algorithmen}
%TODO

\subsubsection{Verifikation}
%TODO

\subsubsection{Analyse}
%TODO

