\section{Theorie}
%TODO @Tobias bitte überprüfe nochmal Begrifflichkeiten
Im Prinzip können Petri-Netze als Graphen aufgefasst werden.
Diese Graphen besitzen allerdings besondere Knoten, die meist als Komponenten bezeichnet werden.
Im folgenden sollen die Komponenten eines Petri-Netzes beschrieben werden.

\paragraph{Eine Transition} ist eine Komponente eines Petri-Netzes.
Sie beschreiben eine elementaren Aktion, die Dinge erzeugen, transportieren, verändern oder vernichten kann.
Eine Transition durch eines Quadrat dargestellt:
\begin{center}
    \begin{tikzpicture}
        \node[transition]{};
    \end{tikzpicture}
\end{center}

\paragraph{Plätze} sind Komponenten an den Objekte, wie z.B. Werkzeuge, Materialien, Produkte oder auch Rezepte gelagert werden können.
Sie werden immer hauptsächlich zum lagern, speichern, sichtbar machen, Zustände repräsentieren verwendet.
Ein Platz wird durch einen Kreis dargestellt:
\begin{center}
    \begin{tikzpicture}
        \node[place](emptyPlace){};
        \node[below=0.2cm of emptyPlace](emptyPlaceLabel){Leerer Platz};
        \node[right= of emptyPlaceLabel](filledPlaceLabel){Gefüllter Platz};
        \node[place, tokens=1, above=0.2cm of filledPlaceLabel](filledPlace){};
    \end{tikzpicture}
\end{center}



\subsection{Variationen}
%TODO

\subsection{Algorithmen}
%TODO

\subsubsection{Verifikation}
%TODO

\subsubsection{Analyse}
%TODO

