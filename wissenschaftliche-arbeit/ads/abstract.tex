%!TEX root = ../dokumentation.tex

\pagestyle{empty}

\iflang{de}{%
% Dieser deutsche Teil wird nur angezeigt, wenn die Sprache auf Deutsch eingestellt ist.
\renewcommand{\abstractname}{Zusammenfassung} % Text für Überschrift
\begin{abstract}
Diese Proseminar-Arbeit führt in die Thematik der Petrinetze ein. Petrinetze sind ein Formalismus zur Beschreibung vernetzter Systeme. Ihre Stärken liegen besonders bei der Darstellung nebenläufiger Prozesse. Transitionen und Plätze sind die Komponenten eines Petrinetzes. Während Transitionen die elementaren Aktionen des modellierten Systems abbilden, können Plätze Objekte oder Informationen speichern beziehungsweise lagern. Nachdem dem Leser die grundlegende Funktionsweise von Petrinetzen bekannt ist, wird auf die Bestimmung von Platzinvarianten als exemplarische Analysemethode eingegangen. Aus Platzinvarianten lassen sich Gleichungen ableiten, die zur weiteren Analyse oder zur Verifikation des Modells genutzt werden können. Anschließend wird das vermittelte Wissen anhand einer Fallstudie noch einmal in ein größeres Beispiel gefasst.
\end{abstract}
}


\iflang{en}{%
% Dieser englische Teil wird nur angezeigt, wenn die Sprache auf Englisch eingestellt ist.
\renewcommand{\abstractname}{Summary} % Text für Überschrift
\begin{abstract}
An abstract is a brief summary of a research article, thesis, review,
conference proceeding or any in-depth analysis of a particular subject
or discipline, and is often used to help the reader quickly ascertain
the paper's purpose. When used, an abstract always appears at the
beginning of a manuscript, acting as the point-of-entry for any given
scientific paper or patent application. Abstracting and indexing
services for various academic disciplines are aimed at compiling a
body of literature for that particular subject.

The terms précis or synopsis are used in some publications to refer to
the same thing that other publications might call an "abstract". In
management reports, an executive summary usually contains more
information (and often more sensitive information) than the abstract
does.

Quelle: \url{http://en.wikipedia.org/wiki/Abstract_(summary)}

\end{abstract}
}