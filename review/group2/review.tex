\documentclass{article}
\usepackage[utf8]{inputenc}
\usepackage[german]{babel}
%\usepackage{tabularx}
\usepackage{tikz}
\usetikzlibrary{positioning, shapes.geometric, arrows}
\usepackage{hyperref}
\hypersetup{colorlinks,urlcolor=[rgb]{0,0,0.5},linkcolor=[rgb]{0,0.3,0.5}}
\usepackage{a4wide}
\usepackage{listings}
\usepackage{csquotes}
\usepackage{ltablex}
%\lstset{language=C}

\author{Arwed Mett}
\title{Review-Vorlage}

\begin{document}
\maketitle
\begin{center}
\begin{tabularx}{5cm}{ll}
    Reviewer: & Arwed Mett\\
    Titel des zu rev. Arbeit & Industry 4.0\
\end{tabularx}
\vspace{1cm}
\end{center}
\begin{tabularx}{\textwidth}{|l|X|X|}
    \hline
    \textbf{Wo?} & \textbf{Was?} & \textbf{Vorschlag}\\
    \hline
    Seite 4 & the economy have & the economy has\\
    \hline
    Seite 4 & steam engines and other machines & Welche andern Maschinen?\\
    \hline
    Seite 4 & the working processes & working processes\\
    \hline
    Seite 4 & the economics & the economy $\rightarrow$ ich weiß nicht ob economy eine mehrzahl hat.\\
    \hline
    Seite 4 & some economic historians & \enquote{some} ist ziemlich waage\\
    \hline
    Seite 4 & two Revolutions & two revolutions\\
    \hline
    Seite 4 &
    During these two Revolutions the industry used more electrical energy and
    the digitalisation took place. & Wann waren denn diese Revolutionen\\
    \hline
    Seite 4 & by the German speaking areas. & Hört sich nicht richtig an.\\
    \hline
    Seite 4 & Germany puts itself in a special situation as it wants to be in the
    leading position in this new kind of technical advanced movement. & Vielleicht begründen oder Quelle angeben\\
    \hline
    Seite 4-5 &
    To sum it up, industry 4.0 is a marketing concept, which describes a
    future process project, in which working processes will get more individualised and customers, as well as, business partners get included &
    Ist es jetzt ein revolutionärer Prozess oder ein Marketingkonzept (Irgendwie passt das nicht zu dem Text darüber)\\
    \hline
    Seite 5 & As any completed Revolution & Ist die Revolution denn schon abgeschlossen? Oben habt ihr geschrieben, sie sei aktuell. Wieso sollte außerdem jede Revolution mehrere Ursachen haben?\\
    \hline
    Seite 5 & has trigger from & was triggered by\\
    \hline
    Seite 5 & Hence, the companies & Hört sich komisch an\\
    \hline
    Seite 5 & Therefore, a second trigger is the increasing need to develop new products. & Irgenwie passt das therefor nicht\\
    \hline
    Seite 5 & to satisfy the customer’s demands with new requirements. & to satisfy the demands of their customers\\
    \hline
    Seite 5 & The prices of all kind of resources & Resource become more and more expensive and limited \ldots \\
    \hline
    Seite 5-6 & The concept of Industry 4.0 is used in a wide range of branches. Especially, it will be
    used in the production of goods, on the one hand, to keep the production costs as low as possible. And on the other hand, to stay competitive. &
    Irgendwie ist an diesem Punkt nicht klar was die Fabrik 4.0 eigentlich ist. Das Thema sollte in der Einleitung vielleicht nicht nur historisch beschrieben werden. Es fehlt eine prezise Beschreibung, was eine Fabrik 4.0 eigentlich ist. Deshalb versteht man die Argumentation nicht. Außerdem macht das mit dem competitive keinen Sinn. Welchen Vorteil bietet die Fabrik 4.0, wodurch eine Fertigungsanlage effizienter bzw. effektiver arbeitet, wodurch ein Unternehmen \enquote{Compatitive} wird?\\

    \hline
    Focus Germany & The main goal of the project & Welches Projekt? Eines der Bundesregierung?\\
    \hline
    Seite 6 & The machines will then document their own
performance and will even be able to initiate maintenance themselves. & The machines will then document their own
performance and will be able to initiate maintenance by themselves, if a malfunction occures.\\
    \hline
    Seite 6 & That is
    why it is vital, to maintain our leading role in the market. & That is
    why it is vital for Germany, to constantly improve their industry. Generell müsst ihr darauf achten neutral zu bleiben.\\
    \hline
    Seite 6 & As the employment of
    technology increases at a rapid pace, especially in developing countries like Bangladesh
    and Indonesia, its importance in production is also rising quickly. &
    Da würde ich eine Statistik zitieren, denn das klingt unglaubwürdig.\\
    \hline
    Seite 6 & energy transition.& Sollte näher erklärt werden, was damit gemeint ist. Bzw. man merkt nicht das ihr das im nächsten Satz erklärt.\\
    \hline
    Seite 6 & This means that German companies should be
    sponsored to take part in this field of development. & Naja nicht unbedingt. Die Meinung muss noch begründet werden. Aus einer liberalen Sichtweise ist dies z.B. nicht unbedingt positiv.\\
    \hline
    Seite 6 & The main driving force behind the project is an organisational platform, as a joint
    venture between ZVEI, VDMA and BITKOM. ZVEI, the German Electrical and Electronic
    Manufacturer’s Association is a union of electrical engineers and the electronic \ldots& Das sollte früher kommen.\\
    \hline
    Seite 6 & who will be in constant contact & who will be constantly in contact.
    Ich dachte eigentlich die Fabrik 4.0 ist vollkommen automatisiert?\\
    \hline
    Seite 7 & will have competitors
    from all over the world & Gibt es noch keine ausländischen Konkurrenten?\\
    \hline
    Seite 7 & to produces & to produce\\
    \hline
    Seite 7 & However, the
    today’s technologies and machines do not allow to produces under unpredictable
    requirements fluctuations. & Der Satz hört sich grammitikalisch nicht korrekt an.\\
    \hline
    Seite 7 & from all kind of & from all kinds of\\
    \hline
    Seite 7 & Thus, the realisation is only capable & Thus, the realisation is only possible\\
    \hline
    Seite 7 & is enormous important to ensure
    Germany’s leading position. & Geht es nicht eher um die Entwicklung?\\
    \hline
    Seite 8 & The main goal of Industry 4.0 is to develop the core system science needed to engineer
complex cyber-physical systems which people can use or interact with. & The main goal of Industry 4.0 is to develop the core system architecture which provides an interface for humans. Ein Vergleich wäre sinnvoll. Sonst versteht man nicht was mit einem CPS eigentlich gemeint ist.\\
    \hline
    Kapitel 4 & & Das sollte vorher kommen. Also die ganze Erklärung von einem CBS. Das Kapitel erklärt die meisten Fragen, die ich oben hatte :D\\ 
    \hline
    
    


\end{tabularx}
\end{document}
