\documentclass{article}
\usepackage[utf8]{inputenc}

\title{Plaung zur Wissenschaftlichen Arbeit über Petri-Netze}
\author{Arwed Mett}

\begin{document}
\maketitle

\section{Allgemein}
Wir werden unsere Arbeit in folgende Themenbereiche aufteilen.
\begin{enumerate}
    \item Anwendung und Einsatzgebiete der Petri-Netze $\rightarrow$ Tobias Dorra
    \item Geschichte $\rightarrow$ Arwed Mett
    \item Auswirkungen $\rightarrow$ Dominic Gehrig
    \item Organisationen / Standards
    \item Zukunftsausblick
\end{enumerate}

\section{Technologien}
Zur Erstellung der Aufgabe werden wir folgende Technologien einsetzen.
\begin{description}
    \item[Format der Arbeit] Latex
    \item[CVS - System] eigener git-Server
        \begin{description}
            \item[Adresse] git@mett.ddns.net:/opt/git/petri-netze.git
        \end{description}
\end{description}

\section{Termine und Fristen}
\begin{description}
    \item[Abgabetermin] 29.06.2015
    \item[Inhalt besprechen] 14.06.2015
    \item[Präsentation] 18.06.2015
\end{description}

\section{Offene Aufgaben}
\begin{enumerate}
    \item Recherchieren $\rightarrow$ bis zum 14.06.2015
    \item Titel Festlegen $\rightarrow$ bis zum 14.06.2015
    \item Generell Inhalt festlegen $\rightarrow$ bis zum 15.06.2015
        \begin{itemize}
            \item Struktur Anlegen
            \item Priorisieren
            \item Clustern
            \item Kommentierte Gliederung erstellen
            \item Schreibaufgaben verteilen
        \end{itemize}
    \item schreiben der Arbeit $\rightarrow$ bis zum 20.06.2015
    \item Test lesen $\rightarrow$ bis zum 23.06.2015
    \item Überarbeiten $\rightarrow$ bis zum 26.06.2015
\end{enumerate}
\end{document}
