\documentclass{article}
\usepackage[utf8]{inputenc}
\usepackage[german]{babel}
\usepackage{tabularx}
\usepackage{tikz}
\usetikzlibrary{positioning, shapes.geometric, arrows}
\usepackage{hyperref}
\hypersetup{colorlinks,urlcolor=[rgb]{0,0,0.5},linkcolor=[rgb]{0,0.3,0.5}}
\usepackage{a4wide}
\usepackage{listings}
%\lstset{language=C}

\author{Arwed Mett, Tobias Dorra, Dominik Gehrig}
\title{Kommentierte Gliederung Proseminar Arbeit Petrinetze}

\begin{document}
\maketitle

\section{Allgemeine Daten}
\begin{tabularx}{\textwidth}{|X|X|}
    \hline
    \textbf{Title:} & Modellierung, Analyse und Verifikation von Petrinetzen bei Fertigungsanlagen.\\
    \hline
    \textbf{Authoren:} & Arwed Mett, Tobias Dorra, Dominik Gehrig\\
    \hline
\end{tabularx}
\section{Einleitung}
\textbf{1 Seite}
In der Einleitung wird das Thema Petri-Netze zu erst einmal allgemein beschrieben.
Es soll eine Übersicht der Einsatzgebiete von Petri-Netzen gegeben werden.

\subsection{Entwicklung}
\textbf{3/4 Seite}
Es wird \textbf{kurz} auf die Geschichte der Petrinetze eingegangen.
Es soll der aktuelle Entwicklungsstand der Petrinetze beleuchtet werden.
Des Weiteren soll darauf eingegangen werden, wie sich die Modellierung der Fertigungsanlagen entwickelt hat.
\section{Theorie} 
\textbf{3 Seite}
Es wird die Funktionsweise von Petrinetzen beschrieben.
Dabei werden zunächst die generellen Konzepte (Stellen, Transitionen) erläutert.
Dann wird auf die grafische Darstellung eingegangen.

\subsection{Variationen}
\textbf{1,5 Seiten}
Es werden verschiedene Variationen der ursprünglichen Petrinetze vorgestellt (Gefärbt, mit Zeitkomponente, Fuzzy)
Der Fokus liegt auf ihren Funktionsweisen und Verwendungsmöglichkeiten.
Es wird wenn nötig allerdings auch auf eventuelle Nachteile eingegangen. 

\subsection{Algorithmen}
Es werden Algorithmen zur Verifikation und "`Engpassanalyse"' von Petrinetzen vorgestellt.
Hier müssen wir uns erst noch Einlesen, daher ist der Umfang dieses Kapitels noch unklar. Es könnte sein, dass wir uns auf eins der beiden Unterthemen beschränken.
Die Algorithmen werden nur anschaulich erklärt und \textbf{nicht bewiesen}!
\subsubsection{Verifikation}
\textbf{2 Seiten}
\subsubsection{Analyse}
\textbf{2 Seiten}

\section{Fallstudie}
\textbf{4 Seiten}
Es soll eine Fertigungsanlage für einen Baukasten eines Anemometers modelliert werden.
Anhand dieses Modells sollen die um vorherigen Kapitel vorgestellten Algorithmen angewandt werden.
\subsection{Modellierung}
\subsection{Verifikation}
\subsection{Analyse}

\section{Fazit}
\textbf{1 Seite}
Es werden die wichtigsten Erkenntnisse über die Petrinetze resümiert.
Außerdem soll ein Ausblick in die Zukunft gegeben werden.

\end{document}
