\documentclass{article}
\usepackage[utf8x]{inputenc}
\usepackage{hyperref}

\author{Arwed Mett, Tobias Dorra}
\title{Mögliche Resourcen}
\begin{document}


\maketitle

\section{Allgemeine Literatur}
\begin{itemize}
    \item Allgemein über Petrie Netze:\\
        \url{http://cs.au.dk/cpnets/industrial-use/control-of-systems/#c6596}
    \item Grundlagen, ausführlich, gut beschrieben, mit Beispielen, Weiterentwicklungen, Sinn und Zweck, ...
         \url{http://link.springer.com/article/10.1007/s00287-013-0758-0}
    \item Modellierungstechnik, Analysemethoden, Fallstudien. Noch nicht gelesen, werde ich aber. Das Inhaltsverzeichnis verspricht viele Beispiele und damit einfache Verständlichkeit, aber trotzdem Tiefgang (Analyse von Petrinetzen!)
        \url{http://link.springer.com/book/10.1007/978-3-8348-9708-4}
    \item Abstracte allgemeine Betrachtung von Petri-Netzen\\
        \url{http://cs.au.dk/fileadmin/site_files/cs/research_areas/centers_and_projects/sttt2007.pdf}
    \item Newsletter (Vielleicht findet man interessante Themen)
        \url{http://www.informatik.uni-augsburg.de/pnnl/}
    \item Onlinekurs mit interaktiven Beispielen \\
        \url{https://kik.informatik.fh-dortmund.de//abschlussarbeiten/fuzzyPetriNetze/index.html}
    \item Sehr Umfangreich und bezieht sich sehr Stark auf die Bahn. Allerdings auch sehr verständlich.\\
        \url{http://link.springer.com/chapter/10.1007/978-3-540-48541-4_13(easymotion-prefix)}
    \item Sehr Mathematisch,  Theorieorientiert, enthält Beweise.
        \url{http://link.springer.com/book/10.1007/978-3-540-76971-2}
\end{itemize}

\section{Petri-Netze bei der Bahn}
\begin{itemize}
    \item \url{http://www.researchgate.net/profile/Michael_Meyer_Zu_Hoerste/publication/260250211_Modelling_Functionality_of_Train_Control_Systems_Using_Petri_Nets/links/53f3b8c70cf256ab87b5cb0d.pdf}
    \item \url{http://daimi.au.dk/CPnets/workshop98/papers/hielscher.pdf}\\
        Dieser Bericht / Modell wurde von zwei Studenten verfasst, deren größte schwierigkeit es war den Prozess zu modellieren.
        Man könnte Bezug auf heutige Software nehmen, die diese Aufgabe bewältigen können. (Der Bericht ist von 1998)
    \item Seite mit sehr vielen generellen Informationen:\\
        \url{http://daimi.au.dk/CPnets/proxy.php?url=/CPnets/workshop98/papers/index}
\end{itemize}

\section{Petri-Netze bei Fertigungsanlagen}
\begin{itemize}
    \item \url{http://link.springer.com/article/10.1007/BF01721805} \\
    \item \url{https://www.tu-ilmenau.de/uploads/media/hs_pn_03.pdf} \\
        Eine Hauptseminararbeit, einfach und verständlich gehalten, mit Praxisbeispiel.
\end{itemize}

\end{document}
