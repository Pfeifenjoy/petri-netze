\documentclass{article}
\usepackage[utf8]{inputenc}
\usepackage[german]{babel}
\usepackage{tabularx}
\usepackage{tikz}
\usetikzlibrary{positioning, shapes.geometric, arrows}
\usepackage{hyperref}
\hypersetup{colorlinks,urlcolor=[rgb]{0,0,0.5},linkcolor=[rgb]{0,0.3,0.5}}
\usepackage{a4wide}
\usepackage{listings}
%\lstset{language=C}

\author{Arwed Mett, Tobias Dorra, Dominik Gehrig}
\title{Strukturplan Proseminar Arbeit Petri-Netze}

\begin{document}
\maketitle

\section{Allgemeine Daten}
\begin{tabularx}{\textwidth}{|X|X|}
    \hline
    \textbf{Title:} & ?\\
    \hline
\end{tabularx}
\section{Einleitung}
In der Einleitung wird das Thema Petri-Netze zu erst einmal allgemein beschrieben.


\subsection{Entwicklung}

\section{Theorie}
Vorstellung und Unterschiede verschiedener Typen von Petri-Netzen.

\section{neue Algorithmen}

\section{Einsatzgebiete}

\section{Beispiele}

\end{document}
